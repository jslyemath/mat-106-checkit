\ProvidesPackage{skillcheckpoints}

\usepackage[T1]{fontenc}

% More colors and names
\usepackage[dvipsnames]{xcolor}

% Skill Checkpoint main color theme
\definecolor{scCOLOR}{HTML}{145da0}
\definecolor{darkergray}{HTML}{404040}

% For setting keys and values
\usepackage{pgfkeys}

% Use to import thru paths
\usepackage{import}

%Header Material
\makeatletter
\newcommand{\acadclass}[1]{\def\@acadclass{#1}}
\newcommand{\tbsection}[1]{\def\@tbsection{#1}}
\newcommand{\duedate}[1]{\def\@duedate{#1}}
\makeatother

%Call Header Material
\makeatletter
\newcommand{\theacadclass}{\@acadclass}
\newcommand{\thetbsection}{\@tbsection}
\newcommand{\theduedate}{\@duedate}
\makeatother

%font
\usepackage{ClearSans}
\usepackage{mathpazo}

\usepackage{parskip}

%margins
\usepackage[margin=0.75in, top=1.5cm, bottom=0.8cm, includeheadfoot]{geometry}

%If you need doublespacing
\usepackage{setspace}

%Get Text Size for Some New Commands
\newlength{\myts}
\makeatletter
\setlength{\myts}{\f@size pt}
\makeatother

% Force \include{} to throw error if file is missing
\makeatletter
\def\mkfilename#1{%
  \if\relax\detokenize\expandafter{#1}\relax\else#1/\fi}
\AddToHook{include/before}%
  {\IfFileExists{\mkfilename\CurrentFilePath\CurrentFile}{}
     {\GenericError{}{Error: File \mkfilename\CurrentFilePath\CurrentFile.tex not found}{\@gobble}{}}}
\makeatother

% Better tables
\usepackage{tabularx,colortbl}
% Even better tables
\usepackage{tabularray}

% Wrap figures
\usepackage{wrapfig}

% Answers command for keys
\usepackage{ifthen}
\newboolean{anstoggle}
\setboolean{anstoggle}{false}
\newcommand{\ans}[1]{\ifthenelse{\boolean{anstoggle}}{\textcolor{scCOLOR}{#1}}{}}
\newenvironment{ansenv}{\ifthenelse{\boolean{anstoggle}}{\color{scCOLOR}}{\expandafter\comment}}{\ifthenelse{\boolean{anstoggle}}{}{\expandafter\endcomment}}

% Custom format for T/F questions
%\newcommand{\tfleft}[1]{\textbf{TRUE}\hspace{5pt} / \hspace{5pt}\textbf{FALSE} \hspace{7pt}  \begin{minipage}[t]{.72\textwidth} #1\end{minipage}}
\setlength{\fboxrule}{1.2pt}
\NewDocumentCommand{\tfleft}{o m}{% #1 is the optional answer ("True" or "False"), #2 is the statement
    \textbf{%
        \ifthenelse{\boolean{anstoggle} \AND \equal{\IfNoValueTF{#1}{dummy}{#1}}{True}}%
            {\fcolorbox{scCOLOR}{white}{TRUE}}{TRUE}%
    }
    \hspace{5pt} / \hspace{5pt}
    \textbf{%
        \ifthenelse{\boolean{anstoggle} \AND \equal{\IfNoValueTF{#1}{dummy}{#1}}{False}}%
            {\fcolorbox{scCOLOR}{white}{FALSE}}{FALSE}%
    }
    \hspace{7pt}
    \begin{minipage}[t]{.72\textwidth} #2 \end{minipage}%
}


% Custom format for general mini-columns (usually within each enumerate item)
\NewDocumentCommand{\minicol}{O{.475} O{.475} m m}{%
  \begin{minipage}[t]{#1\textwidth} #3 \end{minipage}%
  \hfill%
  \begin{minipage}[t]{#2\textwidth} #4 \end{minipage}%
}

% Custom underlined space command
\NewDocumentCommand{\fillinblank}{O{2.25in}m}{%
  \makebox[#1]{\underline{\hspace{0pt}\makebox[#1][c]{\ans{#2}}}}%
}

% Math packages
\usepackage{amssymb, amsmath, amsthm, bm, textcomp, gensymb}

%Nice shortcuts for set names and other math tools
\newcommand{\Q}{\mathbb{Q}}
\newcommand{\Z}{\mathbb{Z}}
\newcommand{\R}{\mathbb{R}}
\newcommand{\C}{\mathbb{C}}
\renewcommand{\Re}{\operatorname{Re}}
\renewcommand{\Im}{\operatorname{Im}}
\DeclareMathOperator{\Arg}{Arg}
\newcommand{\dlim}{\lim\limits}
\newcommand{\dint}{\displaystyle\int}
\DeclareMathSymbol{\shortminus}{\mathbin}{AMSa}{"39}
\usepackage{physunits}
\DeclareRobustCommand{\yd}{\ensuremath{%
\expandafter\units@separator\mathrm{yd}}}
\renewcommand{\N}{\mathbb{N}}

% arcs command for geometry statements
\makeatletter
\DeclareFontFamily{U}{tipa}{}
\DeclareFontShape{U}{tipa}{m}{n}{<->tipa10}{}
\newcommand{\arc@char}{{\usefont{U}{tipa}{m}{n}\symbol{62}}}%

\newcommand{\arc}[1]{\mathpalette\arc@arc{#1}}

\newcommand{\arc@arc}[2]{%
  \sbox0{$\m@th#1#2$}%
  \vbox{
    \hbox{\resizebox{\wd0}{\height}{\arc@char}}
    \nointerlineskip
    \box0
  }%
}
\makeatother

%Needed for some extra graphical tools
\usepackage{graphicx}
\usepackage{soul}
\newcommand{\hlc}[2][yellow]{{\sethlcolor{#1}\hl{#2}}}
\usepackage[export]{adjustbox}



\usepackage{ulem}

%Produces beautiful boxes. I copied some code here, so I don't know what the [most] option does
\usepackage[most]{tcolorbox}


%More options back-end coding of new functions
\usepackage{xparse}
\usepackage{xpatch}

%Defining our new definition box. Still working out how to get extra padding around it and inside of it (just at the top), without needing \begin{figure} and an extra padding option every time in the main document
\newtcolorbox{defBox}[3][]{
arc=2mm,
lower separated=false,
fonttitle=\bfseries,
colbacktitle=green!50!black!8!white,
coltitle=black,
enhanced,
attach boxed title to top left={xshift=0.5cm,
        yshift=-2mm},
colframe=green!35!black,
colback=green!50!black!8!white,
overlay={
\node[draw=green!35!black,very thick,
inner xsep=3.5mm,
fill= green!50!black!8!white,rounded corners=1mm,
yshift=3pt,
xshift=-.5cm,
left,
text=black,
anchor=east,
font=\bfseries]
at (frame.north east) {#3};},
title=#2,#1}


\newtcolorbox{thmBox}[3][]{
arc=2mm,
lower separated=false,
fonttitle=\bfseries,
colbacktitle=blue!50!black!8!white,
coltitle=black,
enhanced,
attach boxed title to top left={xshift=0.5cm,
        yshift=-2mm},
colframe=blue!35!black,
colback=blue!50!black!8!white,
         overlay={
\node[draw=blue!35!black,very thick,
inner xsep=3.5mm,
fill= blue!50!black!8!white,rounded corners=1mm,
yshift=3pt,
xshift=-.5cm,
left,
text=black,
anchor=east,
font=\bfseries]
at (frame.north east) {#3};},
title=#2,#1}


\newtcolorbox{noteBox}[2][]{
arc=2mm,
lower separated=false,
fonttitle=\bfseries,
coltitle=black,
colbacktitle=black!4!white,
enhanced,
colback=black!4!white,
attach boxed title to top left={xshift=0.5cm,
        yshift=-2mm},
title=#2,#1}

% Very specific theorem-like environments from my notes, but rarely used in my quizzes
\newenvironment{theorem}[1][]{%
    \begin{figure}[h!]
    \begin{thmBox}[top=3.5mm]{Theorem}{#1}%
}{%
    \end{thmBox}
    \end{figure}%
    \vspace*{-0.91\myts}
}

\newenvironment{axiom}[1][]{%
    \begin{figure}[h!]
    \begin{thmBox}[top=3.5mm]{Axiom}{#1}%
}{%
    \end{thmBox}
    \end{figure}%
    \vspace*{-0.91\myts}
}

\newenvironment{definition}[1][]{%
    \begin{figure}[h!]
    \begin{defBox}[top=3.5mm]{Definition}{#1}%
}{%
    \end{defBox}
    \end{figure}%
    \vspace*{-0.91\myts}
}

\newenvironment{statement}[1][]{%
    \begin{figure}[h!]
    \begin{noteBox}[top=3.5mm]{#1}%
}{%
    \end{noteBox}
    \end{figure}%
    \vspace*{-0.91\myts}
}

\newenvironment{property}[1][]{%
    \begin{figure}[h!]
    \begin{thmBox}[top=3.5mm]{Property}{#1}%
}{%
    \end{thmBox}
    \end{figure}%
    \vspace*{-0.91\myts}
}

%need these for some of the nifty tikz commands
\usepackage{tikz}
\usepackage{tkz-euclide}
\usepackage{tikz-cd}
\usetikzlibrary{calc,trees,positioning,arrows,fit,shapes,calc,angles,quotes,backgrounds,datavisualization,datavisualization.formats.functions,math,intersections}
\tikzstyle{aligntotop}=[baseline={([yshift={-\ht\strutbox}]current bounding box.north)},outer sep=0pt,inner sep=0pt]

\tikzmath{
    \xa = -5; \xz = 5;
    \ya = -5; \yz = 5;
    \xs = 1; \ys = 1;
    \xn = \xa + 1; \yn = \ya+1;
    \xm1 = \xa - \xs; \xm2 = \xz + \xs;
    \ym1 = \ya - \ys; \ym2 = \yz + \ys;
    }

\newcommand{\setmyx}[3][1]{\tikzmath{
    \xa = #2; \xz = #3;
    \xs = #1;
    \xn = \xa + 1;
    \xm1 = \xa - \xs; \xm2 = \xz + \xs;
    }}

\newcommand{\setmyy}[3][1]{\tikzmath{
    \ya = #2; \yz = #3;
    \ys = #1;
    \yn = \ya + 1;
    \ym1 = \ya - \ys; \ym2 = \yz + \ys;
    }}

\usepackage{pgfplots}

\pgfplotsset{compat=newest}

\tikzset{>={Stealth[length=8pt,width=7pt]}}
\pgfplotsset{bluesoldot/.style={color=blue,only marks,mark=*}}
\pgfplotsset{blueholdot/.style={color=blue,fill=white,only marks,mark=*}}
\pgfplotsset{slydes axis style/.style ={
  axis background/.style = {fill=white},
  thick,
  axis x line=center,
  axis y line=center,
  grid = both,
  xticklabel style = {font=\tiny,yshift=0.5ex},
  yticklabel style = {font=\tiny,xshift=0.5ex},
  xlabel style={font=\tiny, xshift=-1.75ex},
  ylabel style={font=\tiny, yshift=-1.75ex},
  xtick={\xa,\xn,...,\xz},
  ytick={\ya,\yn,...,\yz},
  every major tick/.append style={thick, major tick length=5pt, black},
  xlabel={$x$},
  ylabel={$y$},
  xlabel style={below right},
  ylabel style={above left},
  xmin=\xm1,
  xmax=\xm2,
  ymin=\ym1,
  ymax=\ym2}}
  \pgfplotsset{printed axis style/.style ={
  axis background/.style = {fill=white},
  thick,
  axis x line=center,
  axis y line=center,
  grid = both,
  xticklabel style = {font=\tiny,yshift=0.5ex},
  yticklabel style = {font=\tiny,xshift=0.5ex},
  xlabel style={font=\tiny, xshift=-1.75ex},
  ylabel style={font=\tiny, yshift=-1.75ex},
  xtick={\xa,\xn,...,\xz},
  ytick={\ya,\yn,...,\yz},
  every major tick/.append style={thick, major tick length=5pt, black},
  xlabel={$x$},
  ylabel={$y$},
  xlabel style={below right},
  ylabel style={above left},
  xmin=\xm1,
  xmax=\xm2,
  ymin=\ym1,
  ymax=\ym2}}

\pgfplotsset{number line style/.style ={
  axis background/.style = {fill=none},
  thick,
  yscale=0.1,
  axis x line=center,
  axis y line=none,
  axis line style={Stealth-Stealth},
  xticklabel style = {font=\tiny,yshift=0.5ex},
  xlabel style={font=\tiny, xshift=-1.75ex},
  xtick={\xa,\xn,...,\xz},
  every major tick/.append style={thick, major tick length=50pt, black},
  xlabel style={below right},
  xmin=\xm1-0.5,
  xmax=\xm2+0.5,
  ymin=-0.5,
  ymax=0.5}}

\pgfplotsset{blank number line style/.style ={
  axis background/.style = {fill=none},
  thick,
  yscale=0.1,
  axis x line=center,
  axis y line=none,
  axis line style={Stealth-Stealth},
  xticklabel style = {font=\tiny,yshift=0.5ex},
  xtick={\xa,\xn,...,\xz},
  every major tick/.append style={thick, major tick length=50pt, black},
  xticklabels=\empty,
  yticklabels=\empty,
  xmin=\xm1-0.5,
  xmax=\xm2+0.5,
  ymin=-0.5,
  ymax=0.5}}

\pgfplotsset{blank axis style/.style ={
  axis background/.style = {fill=white},
  thick,
  yscale=0.1,
  axis x line=center,
  axis y line=center,
  grid = both,
%   xticklabel style = {font=\tiny,yshift=0.5ex},
%   yticklabel style = {font=\tiny,xshift=0.5ex},
%   xlabel style={font=\tiny, xshift=-1.75ex},
%   ylabel style={font=\tiny, yshift=-1.75ex},
  xticklabels=\empty,
  yticklabels=\empty,
  xtick={\xa,\xn,...,\xz},
  ytick={\ya,\yn,...,\yz},
  every major tick/.append style={thick, major tick length=5pt, black},
%   xlabel={$x$},
%   ylabel={$y$},
%   xlabel style={below right},
%   ylabel style={above left},
  xmin=\xm1,
  xmax=\xm2,
  ymin=\ym1,
  ymax=\ym2}}

% For the rounded checkbox at on checkpoints
\usepackage{xspace}
\newcommand{\roundbox}[1]{\xspace\begin{tikzpicture}[baseline=-0.7ex]
   \node[draw, fill=white,thick,rectangle, rounded corners, minimum size=#1] {};
\end{tikzpicture}\xspace}

%Better options for lists and enumerates
\usepackage[shortlabels]{enumitem}
\setlist[enumerate]{label=(\alph*)}
\usepackage{multicol}

% Makes a better columns option for tables. I forget I made this half the time.
\usepackage{array}
\newcolumntype{C}[1]{>{\centering\arraybackslash}m{#1}}
\newcolumntype{Y}{>{\centering\arraybackslash}X}
\renewcommand{\tabularxcolumn}[1]{m{#1}}

% Set up title
\setcounter{secnumdepth}{-1}
\usepackage{titlesec}
\titleformat{\section}
  {\large\sffamily\bfseries}{\thesection}{em}{}

\titlespacing{\section}{0pt}{0pt}{-2pt}


% version macro
\newcommand{\vseed}{}
\newcommand{\setvseed}[1]{\renewcommand{\vseed}{#1}}

% Skill macros
% initialise currentskill macro
\newcommand\currentskill{}

\newtcolorbox{skill}[1]{before skip balanced=0pt,after skip balanced=12pt, boxrule=0.7mm, colback=scCOLOR!15!white, colframe=scCOLOR,title=\protect\section{#1}, fontupper=\sffamily\color{darkergray}, fonttitle=\color{white}}

% command for defining macro globally
\newcommand{\gdefcurrentskill}[1]
  {\gdef\currentskill{#1}}

\newenvironment{skillbox}[1]{\gdefcurrentskill{#1}\begin{skill}{\currentskill}}{\end{skill}}

\newcommand{\setskilldesc}[2]{\pgfkeyssetvalue{/skills_dict/#1}{#2}}
\newcommand{\getskilldesc}[1]{\pgfkeysvalueof{/skills_dict/#1}}

\input{Skill Descriptions.tex}

% Currently the preferred method, as it automatically pulls from Skill Descriptions.tex
\newcommand{\skillheader}[1]{\begin{skillbox}{#1} \getskilldesc{#1} \end{skillbox}}

% For compatibility with older skill files
\newenvironment{setskill}[1]{\gdefcurrentskill{#1}\begin{skill}{\currentskill}}{\end{skill}}

% Adds a newpage to separate skills
\newcommand{\skillpage}[1]{\input{#1}\newpage}

%insert nicer urls if needed in online PDF checkpoints (pandemic era)
\usepackage{hyperref}
\hypersetup{colorlinks=true, urlcolor=scCOLOR,linkcolor=scCOLOR}
\usepackage{url}

%Change \maketitle to activity style
\makeatletter
\renewcommand\maketitle{\sffamily\textit{\color{darkgray}\@title}}
\makeatother

% header/footer/page numbers
\usepackage{fancyhdr}
\setlength{\headheight}{22pt}
\setlength{\headsep}{10pt}
\renewcommand{\headrulewidth}{0pt}
\pagestyle{fancy}
\usepackage[bottom]{footmisc}
\fancyhead[C]{}
\makeatletter
\fancyhead[R]{\sffamily\color{darkgray} \@acadclass{} \hspace{5pt} \textbullet{}\hspace{5pt}  \@date \hspace{5pt} \textbullet{}\hspace{5pt}  \@author}
\makeatother
\fancyhead[L]{\sffamily\color{darkgray} Name: \underline{\Large\phantom{Blank}\hspace{2.9in}}}
\fancyfoot[RE, LO]{\maketitle}
\fancyfoot[C]{\sffamily\textit{\color{darkgray}v\vseed}}
\fancyfoot[LE, RO]{\sffamily\textit{\color{darkgray}\currentskill\hspace{0.5em}\roundbox{20pt}}}

\newcommand{\cacadclass}{
\makeatletter
\@acadclass{}
\makeatother
}

\newcommand{\setname}[1]{%
    \fancyhead[L]{\sffamily\color{darkgray} Name: %
    \ifthenelse{\equal{#1}{Blank} \OR \equal{#1}{blank}}%
        {\underline{\Large\phantom{Blank}\hspace{2.9in}}}%
        {\Large #1}}%
    \pdfbookmark[0]{\texorpdfstring{#1}{#1}}{#1}%
}

\makeatletter
\newcommand{\setsect}[1]{%
    \fancyhead[R]{\sffamily\color{darkgray}
    \@acadclass{}%
    \ifthenelse{\equal{#1}{Blank} \OR \equal{#1}{blank}}%
    {}% Do nothing if #1 is Blank or blank
    {-#1}% Otherwise, append -#1
    \hspace{4pt} \textbullet{}\hspace{4pt} \@date \hspace{4pt} \textbullet{}\hspace{4pt} \@author}%
}
\makeatother

% Roman Numerals
\newcommand{\rnc}[1]{\MakeUppercase{\romannumeral #1}}
\newcommand{\mathrnc}[1]{\mathrm{\MakeUppercase{\romannumeral #1}}}

\usepackage{xspace}

% Egyptian Hieroglphics
\usepackage{hieroglf}

% "Babylonian" Numerals
\newcommand{\babz}{\tikz{\draw[rounded corners=0.1mm, fill=black] (0,0) -- (0.2,0) -- (0.1,0.2) -- cycle;
\draw[rounded corners=0.15mm, fill=black] (0,0.2) -- (0.2,0.2) -- (0.1,0.4) -- cycle;}\xspace}
\newcommand{\babo}{\tikz{\draw[rounded corners=0.15mm, fill=black] (0,0.4) -- (0.1,0) -- (0.2,0.4) -- cycle;}\xspace}
\newcommand{\babt}{\tikz{\draw[rounded corners=0.2mm, fill=black] (0.4,0.4) -- (0.05,0.2) -- (0.4,0) -- (0.25,0.2) -- cycle;}\xspace}

% Multiplication Tables
% https://tex.stackexchange.com/questions/210670/addition-and-multiplication-tables
\newcolumntype{O}{>{\centering\arraybackslash}p{1.5em}}

\ExplSyntaxOn
\NewDocumentCommand{\additiontable}{m}
 {% #1 is the base
  \induktio_make_table:nnn { #1 } { + } { $+$ }
 }
\NewDocumentCommand{\multiplicationtable}{m}
 {% #1 is the base
  \induktio_make_table:nnn { #1 } { * } { $\times$ }
 }

\tl_new:N \l__induktio_table_body_tl
\int_new:N \l__induktio_row_number_int
\cs_new_protected:Npn \induktio_make_table:nnn #1 #2 #3
 {
  \group_begin:
  \dim_set:Nn \doublerulesep { 0pt }
  \int_zero:N \l__induktio_row_number_int
  % add the operation symbol
  \tl_set:Nn \l__induktio_table_body_tl { #3 }
  % add the first row
  \int_step_inline:nnnn { 0 } { 1 } { #1-1 }
   { \tl_put_right:Nx \l__induktio_table_body_tl { & \int_to_Base:nn { ##1 } { #1 } } }
  \tl_put_right:Nn \l__induktio_table_body_tl { \\ \hline\hline }% two rules here
  % add the subsequent rows
  \int_step_inline:nnnn { 0 } { 1 } { #1-1 }
   {
    % first column
    \tl_put_right:Nx \l__induktio_table_body_tl { \int_to_Base:nn { ##1 } { #1 } }
    % subsequent columns
    \int_step_inline:nnnn { 0 } { 1 } { #1-1 }
     {
      \tl_put_right:Nx \l__induktio_table_body_tl
       {
        & \int_to_Base:nn { ####1 #2 \l__induktio_row_number_int } { #1 }
       }
      }
    \tl_put_right:Nn \l__induktio_table_body_tl { \\ \hline }
    \int_incr:N \l__induktio_row_number_int
   }
  % print the table
  \begin{tabular}{c||*{#1}{O|}}% two rules here
  \tl_use:N \l__induktio_table_body_tl
  \end{tabular}
  \group_end:
 }
\ExplSyntaxOff

% For creating trueemptypages when we need to force double-siding with one-siders mixed with two-siders
\newcommand{\trueemptypage}{\clearpage\begingroup\pagestyle{empty}\cleardoublepage\endgroup}