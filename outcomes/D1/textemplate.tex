\setvseed{\VAR{seed}}
\skillheader{D1}

\begin{enumerate}
    \item Consider the number $\VAR{pv_dec_string}$. Identify and name each of the place values used in this number. Write these names using both a numerical label (e.g.\ ``10s'') and an English-word label (e.g.\ ``tens'').

    \vspace{20pt}
    \begin{ansenv}
        \VAR{pv_ans_text}
    \end{ansenv}
    \vfill

    \item Using \VAR{units_block_choice} to represent the units, draw a base-ten block representation of the number \VAR{blocks_dec}.

    \vspace{20pt}
    \begin{ansenv}
        \VAR{blocks_ans_text}
    \end{ansenv}
    \vfill
\end{enumerate}

\newpage

\skillheader{D1-E}

\begin{enumerate}
    \item If one $\VAR{units_block_choice}$ represents one unit, which type of block represents one tenth? Why?

    \vspace{20pt}
    \begin{ansenv}
        It takes ten tenths to create a single unit. Likewise, it takes ten \VAR{tenths_block}s to create a single \VAR{units_block}. Thus, a \VAR{tenths_block} represents one tenth.
    \end{ansenv}
    \vfill

    \item If one $\VAR{units_block_choice}$ represents one unit, which type of block represents one hundredth? Why?

    \vspace{20pt}
    \begin{ansenv}
        It takes one hundred hundredths to create a single unit. Likewise, it takes one hundred \VAR{hundredths_block}s to create a single \VAR{units_block}. Thus, a \VAR{hundredths_block} represents one hundredth. 
    \end{ansenv}
    \vfill

    \item How do base-ten blocks physically demonstrate how a base-ten place value system works?

    \vspace{20pt}
    \begin{ansenv}
        In a base-ten place value system, each place value must be ten times the place value to its right. It takes ten units to make a long, ten longs to make a flat, and ten flats to make a large cube. Thus, the four types of base ten blocks can be used for any 4 consecutive base-ten place values.
    \end{ansenv}
    \vfill

\end{enumerate}
