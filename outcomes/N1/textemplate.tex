\setvseed{\VAR{seed}}
\skillheader{N1}

Determine if each of the statements below is true or false. Circle each answer.\\

\begin{enumerate}

    \item \tfleft[\VAR{p1_ans}]{\VAR{p1_prob}}\\[4ex]

    \item \tfleft[\VAR{p2_ans}]{\VAR{p2_prob}}\\[4ex]

    \item \tfleft[\VAR{p3_ans}]{\VAR{p3_prob}}\\[4ex]

    \item \tfleft[\VAR{p4_ans}]{\VAR{p4_prob}}\\[4ex]

    \item \tfleft[\VAR{p5_ans}]{\VAR{p5_prob}}\\[4ex]

    \item \tfleft[\VAR{p6_ans}]{\VAR{p6_prob}}\\[4ex]

    \item \tfleft[\VAR{p7_ans}]{\VAR{p7_prob}}\\[4ex]

    \item \tfleft[\VAR{p8_ans}]{\VAR{p8_prob}}\\[4ex]

\end{enumerate}

% \newpage

% \skillheader{N1-E}

% Fully explain why each of the following statements from the previous page are true or false. If a statement is \textbf{\textcolor{BrickRed}{false}}, provide a \textbf{\textcolor{BrickRed}{counterexample}}. If it is \textbf{\textcolor{OliveGreen}{true}}, provide a \textbf{\textcolor{OliveGreen}{general explanation}} appropriate for elementary students. Any general explanations must be logically sound, and must be general enough to cover all possible cases (not relying on specific examples).

% \begin{enumerate}

%     \item \VAR{explain_prob_1}

%     \vfill
%     \begin{ansenv}
%         \VAR{explain_ans_1}
%     \end{ansenv}
%     \vfill

%     \item \VAR{explain_prob_2}

%     \vfill
%     \begin{ansenv}
%         \VAR{explain_ans_2}
%     \end{ansenv}
%     \vfill

%     \item \VAR{explain_prob_3}

%     \vfill
%     \begin{ansenv}
%         \VAR{explain_ans_3}
%     \end{ansenv}
%     \vfill
% \end{enumerate}