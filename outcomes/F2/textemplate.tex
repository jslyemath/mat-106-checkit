\setvseed{\VAR{seed}}
\skillheader{F2}

\begin{enumerate}
    \item \VAR{p1_text}$\VAR{p1_math}$.

    \ifthenelse{\equal{\VAR{p1_type}}{line}}{
    \vspace{20pt}
    \setmyx{0}{\VAR{p1_ticks}}
    \begin{tikzpicture}[scale=2.3]
        \begin{axis}[blank number line style,
        extra x ticks={0,\VAR{p1_orig_loc},\VAR{p1_requested_loc}},
        extra x tick labels={0,$\VAR{p1_label_b}$,\ans{$\VAR{p1_label_c}$}}]
            \ans{\node at (axis cs:\VAR{p1_requested_loc},0) {\tikz \fill[scCOLOR] (0,0) circle (2pt);};}
        \end{axis}
    \end{tikzpicture}
    }{\ans{\VAR{p1_ans_text}}}

    \vfill

    \item \VAR{p2_text}$\VAR{p2_math}$.

    \ifthenelse{\equal{\VAR{p2_type}}{line}}{
    \vspace{20pt}
    \setmyx{0}{\VAR{p2_ticks}}
    \begin{tikzpicture}[scale=2.3]
        \begin{axis}[blank number line style,
        extra x ticks={0,\VAR{p2_orig_loc},\VAR{p2_requested_loc}},
        extra x tick labels={0,$\VAR{p2_label_b}$,\ans{$\VAR{p2_label_c}$}}]
            \ans{\node at (axis cs:\VAR{p2_requested_loc},0) {\tikz \fill[scCOLOR] (0,0) circle (2pt);};}
        \end{axis}
    \end{tikzpicture}
    }{\ans{\VAR{p2_ans_text}}}

    \vfill

    \item \VAR{p3_text_math}.

    \ifthenelse{\equal{\VAR{p3_type}}{line}}{
    \vspace{20pt}
    \setmyx{0}{\VAR{p3_ticks}}
    \begin{tikzpicture}[scale=2.3]
        \begin{axis}[blank number line style,
        extra x ticks={0,\VAR{p3_orig_loc},\VAR{p3_requested_loc}},
        extra x tick labels={0,$\VAR{p3_label_b}$,\ans{$\VAR{p3_label_c}$}}]
            \ans{\node at (axis cs:\VAR{p3_requested_loc},0) {\tikz \fill[scCOLOR] (0,0) circle (2pt);};}
        \end{axis}
    \end{tikzpicture}
    }{\ans{\VAR{p3_ans_text}}}

    \vfill
\end{enumerate}

\newpage
\skillheader{F2-E}

Fully explain how your area model for (a) on the previous page represents $\VAR{p1_math}$. Explain, as if to a student, how you knew to take each step in its construction process. Along the way, explain how the model represents the numerator and denominator of the fraction.

